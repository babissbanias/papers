\def\e#1: {%
  \setbox0=\hbox{\bf#1:}%
  \dimen0=5em%
  \advance\dimen0 by -\wd0%
  \par\penalty-200\hangindent=6em\noindent\hskip\dimen0\box0\quad}

\e p2, f1.1, l14: \& should be \&\&
\e p2: The discussion assumes implicitly that indices are over 
  nonnegative integers.
\e p2: ``this proof''---not clear that I talk about invariant's proof
\e p2: ``can be rephrased as follows'' should be ``is''
\e p5: Might need to add {\tt\char`\\null} after C.
\e p6: The core of hol provers is {\it supposed\/} to be small.
\e p7: ``that was exceptions'' should be ``that has exceptions''
\e p12: ``variable'' on lhs should be ``variable-declaration''
\e p13, f2.5: the rule [asgn] should {\it not\/} handle parallel assignments
\e p13: ``$1$, $2$, $3$ \dots'' should be ``$1$, $2$, $3$, \dots''
\e p19: ``log file'' should be ``{\bf l}og {\bf f}ile''
\e p19: ``The result of the whole run is'' should be ``FreeBoogie prints''
\e p21: ``all transformations in FreeBoogie, except loop handling,''
  should be ``all transformations in FreeBoogie are complete, except
  loop handling.''
\e p22, f3.4: Core Boogie does {\it not\/} have parallel assignments.
\e p23, f3.5: ``text'' should be ``{\sl text\/}''
\e p23: ``between generated classes'' should be ``in the generated code''
\e p24: ``Terminal classess, those that have no subclass declared in the
  abstract grammar'' should be ``Terminal classes, which are those without
  subclasses''. Also, remove ``that initializes those fields''.
\e p27: ``The initial call $u_1.u_1(u_2,\ldots,u_n)$''
  should be ``The initial call $u_1.m_1(u_2,\ldots,u_n)$''
\e p30: ``reference-paths'' should be ``reference-path''
\e p30: after ``safeguard against bugs'' add ``in FreeBoogie''
\e p34: ``$1$, $2$, $3$, $\ldots$'' appears with wrong commas twice
\e p35: ``all interpretations'' should be ``all models'', twice (I think!)
\e p38: ``It if then'' should be ``It is then''
\e p45: ``another possible source'' should be ``another source''
\e p47, d13: The definition of an optimization problem is overly complicated.
  It is enough to define a cost function $c:L\times L\to\Re_+$. A solution~$y$
  for an input~$x$ is feasible when $c(x,y)$ is finite.
\e p48 onwards: ``MINS'' should be ``mins''; ``LIS'' should be ``lis''
\e p49, f4.3: ``{\it pred\/}'' should be ``{\it predecessor\/}''
\e p58: ``$r(y)$'' at the end of (f) should be ``$r(x_{n+1})$''
\e p58: After Theorem 1 I should say what $E$~is.
\e p63: Remove ``when $c(x)$ is not a copy node''.
\e p64: ``version optimal'' should be ``version-optimal''
\e p65: ``(1)~restricting the input'' should be ``(1)~restricting the
  search space''
\e p73, e5.20: There is a $b_x$ missing in the second branch.
\e p74, e5.22: The typesetting of the range of $\land$ is messed up.
\e p126: ``the set of booleans $\{\cdots\}$'' should be ``the set $\{\cdots\}$
  of booleans''
\e p128: ``microsoft'' should be ``Microsoft''

\bye
