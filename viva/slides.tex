\documentclass{beamer}
\usepackage[T1]{fontenc}
\usepackage[math]{iwona}
\usepackage{hyperref}
\usepackage{pgfpages}
\usepackage{tikz}
\usepackage{graphicx}

\setbeamertemplate{navigation symbols}{}
\setbeamercolor{normal text}{bg=black,fg=white}
\setbeamercolor{alerted text}{fg=yellow}
\setbeamercolor{structure}{fg=green}

\newcommand{\titletext}{The Design and Algorithms of a Verification Condition
  Generator}
\definecolor{lightblue}{rgb}{0.5,0.5,1}
\hypersetup{colorlinks,linkcolor=lightblue,citecolor=lightblue,urlcolor=lightblue}
\hypersetup{
  pdfauthor={Radu Grigore},
  pdftitle={\titletext}}

\title{\titletext}
\author{Radu Grigore}
\subtitle{FreeBoogie}
\date{23 August 2010}

\def\,#1{{\lineskiplimit=0pt \oalign{\relax#1\crcr\hidewidth,\hidewidth}}}

\begin{document}
\maketitle

% Audience: Rustan and Henry.
% Principles:
%   - give a good overview
%     - spend plenty of time explaining how pieces fit together
%     - mention all the contributions explicitly
%   - go in depth into one subject (say, reachability)
% Plan
%   - motivation
%   - design and overview
%   - 
% Extra slides (for answering questions):
%   - TODO 

%{{{ motivation
\begin{frame}
  \frametitle{program verifiers}
  \begin{block}{goals}
  \begin{itemize}
  \item
    \emph{significantly} improve the quality of programs 
    \emph{without} requiring much work from programmers
  \item 
    check \emph{automatically} that \emph{efficient} programs do
    \emph{what they should}
  \item
    easy to use, fast, \dots
  \end{itemize}
  \end{block}

  \begin{block}{guts}
  \begin{itemize}
  \item compiler-like
  \item programming language theory and algorithms
  \end{itemize}
  \end{block}
\end{frame}

\begin{frame}
\end{frame}
%}}}
%{{{ old
\begin{frame}
  \frametitle{Reachability Analysis for Annotated Code}
  coauthors: Mikol\'a\v{s} Janota and Micha{\l} Moskal\bigskip
  \begin{itemize}
  \item We present a general technique to catch
    \begin{itemize}
    \item \emph{dead code},
    \item bugs (or intended unsoundness) in the VC generator,
    \item inconsistent axiomatizations or specifications, and
    \item assertions that always fail (doomed code).
    \end{itemize}
  \item The technique is simple because
    \begin{itemize}
    \item we rely on a previous stage to cut loops and
    \item we use strongest postcondition (instead of wp).
    \end{itemize}
  \item The naive implementation is unusably slow, as it calls
    the prover once per program point. I sped that up to $<2$
    calls to the prover per method on average, which made it 
    usable.
  \end{itemize}
\end{frame}

\begin{frame}
  \frametitle{Edit and Verify}
  coauthor: Micha{\l} Moskal\bigskip
  \begin{itemize}
  \item Idea: If the program and specs changed a little since
    the last verification, then don't redo all the work!
  \item proof of concept implementation in Fx7
    \begin{itemize}
    \item keep the last prover query
    \item preprocess the new query, using the previous one
    \end{itemize}
  \end{itemize}
\end{frame}

\begin{frame}
  \frametitle{Strongest Postcondition of Unstructured Programs}
  coauthors: Julien Charles, Fintan Fairmichael, Joseph Kiniry
  \bigskip
  \begin{enumerate}
  \item semantics
    \begin{itemize}
    \item correctness of a flowgraph
    \item Hoare triples vs WP vs SP
    \end{itemize}
  \item algorithms
    \begin{itemize}
    \item \emph{definition of passive form}
    \item complexity of finding an optimal passive form
    \end{itemize}
  \end{enumerate}
\end{frame}

\begin{frame}
  \frametitle{FreeBoogie}
  \begin{itemize}
  \item Java clone of MSR's Boogie
  \item platform for some research
  \item status
    \begin{itemize}
    \item after a major change that is not quite finished:
      BoogiePL$\to$Boogie2
    \item main missing parts:
      \begin{itemize}
      \item invariants inference
      \item encoding Boogie2 types
      \end{itemize}
    \end{itemize}
  \item \url{http://code.google.com/p/freeboogie}
  \end{itemize}
\end{frame}
%}}}
%{{{ Q&A
\begin{frame}
\centerline{\Huge Q\&A}
\end{frame}
%}}}
\end{document}

