\documentclass{article} % <<<

\usepackage[T1]{fontenc}
\usepackage[sc]{mathpazo}
\usepackage[english]{babel}
\usepackage{microtype}
\usepackage{rgalg}

\usepackage[colorlinks]{hyperref} % keep it last to avoid some warnings

\linespread{1.05} % for Palatino

\newcounter{propertyCounter}

\newcommand{\sfline}[1]{\\\hbox{\hspace{3em}\textsf{#1}}\\}
\newcommand{\property}{\stepcounter{propertyCounter}\bigskip\noindent\textbf{\Alph{propertyCounter}.}\enskip}

\showboxbreadth=100
\showboxdepth=5
% >>>
\begin{document}
\section{Implementation} % <<<

% - overview
% - ocaml part
%   - parser, static checks (briefly)
%   - instrumenter
%     - inserted bytecode, Barista
%     - inheritance
% - java part
%   - data structures
%   - algorithms (briefly)

Ideally, given a \textit{classpath} and a properties \textit{topl\_1}, \textit{topl\_2}, \dots, the user would instrument the code with one command:
\sfline{toplc \textit{classpath topl\_1 topl\_2 \dots}}
Then all classes in \textit{classpath} signal if a property is broken.
All is not automated yet; however, it helps to keep in mind the goal while reading the rest.

The TOPL compiler (\textsf{toplc}) is written in OCaml.
The TOPL checker is written in Java.

\subsection{Compiler} % <<<

The input of the compiler is a set of properties and a set of classes.
The output of the compiler is a checker and instrumented classes.
The checker is itself a Java class.
The instrumented classes should have almost the same behavior as the given classes:
They are allowed to use more resources.

\paragraph{Desugaring.}
The parser desugars the TOPL language into a simpler, intermediate form.
For example:
(1)~method-name patterns are prefixed according to the \textsf{prefix} directive;
(2)~\textsf{call} and \textsf{return} transitions get one step, while other transitions are desugared into a \textsf{call} followed by a \textsf{return};
(3)~lowercase patterns and constant patterns are turned into guards;
(4)~uppercase patterns are turned into actions.
The parser builds an AST that closely follows the language for which we have formal operational semantics.

\paragraph{Static Checks.}
Next, the compiler checks that the property is well-formed.
For example:
(1)~two parallel writes to automaton variables must have distinct destinations;
(2)~reads from automaton variables must be preceded by writes.

\paragraph{Inheritance.}
Next, the property is transformed so it explicitly identifies Java methods.
Each transition step has a guard, and each guard has a tag component.
A \emph{tag} has a method name pattern, an arity pattern, and an event type pattern.
The latter is either \textsf{call} or \textsf{return}.
For example the tag guard ``\textsf{call *.$\langle$freeboogie.*.eval$\rangle$[2]}'' refers to all the calls to methods with two arguments whose full name matches the pattern ``\textsf{freeboogie.*.eval}'' \emph{and all the methods that override them}.
A tag guard does not determine on its own a set of program points.
A tag guard together with a set of class signatures determine a set of of program points.

Each observable method in the classpath receives two integer identifiers: one for the call event and one for the return event.
An \emph{observable} method is one whose name matches a pattern that appears in an \textsf{observing} directive.
The \textsf{observing} directive is not needed later.
Each tag guard is translated into a set of identifiers.

\textit{Merging}. At this point several properties are merged into a single property that may have multiple start vertices.
Vertices are no longer identified by name.
Start vertices are no longer identified by the string used as their name; instead, start vertices are explicitly enumerated.
Each property used to have a pattern for observable methods;
now each vertex has a set of observable event identifiers.

After these modifications the property is dumped into an easy-to-parse textual format, in the file \textsf{Property.text}.
The initial property may use Java literals.
For example the transition label ``\textsf{*.eval($\langle$34$\rangle$)}'' turns into a guard that that matches on calls to the method \textsf{eval} for which the value of the first argument is~\textsf{34}.
The string from the \textsf{message} directive is also a Java literal.
All such constants are collected into an array \textsf{constants} of \textsf{Object} that is written in the file \textsf{topl/Property.java}.
The file \textsf{Property.text} refers to constants by their index in the array \textsf{constants}.

Ideally, \textsf{Property.java} would be compiled by \textsf{toplc}.
Currently one has to manually invoke \textsf{javac}.

\paragraph{Instrumentation.}
The bytecode of each observable method is instrumented.
The first and last actions performed by the instrumented version are calls to \sfline{topl.Checker.check(event)}
Each \textsf{event} carries an identifier and an array of \textsf{Object}.
The array holds either the arguments, or the return value.

For instrumenting Java bytecode we use a fork of the library Barista.
Our version of Barista handles low-level details such as relative offsets given as a count of bytes.

% >>>
\subsection{Checker} % <<<

The runtime checker of TOPL properties resides entirely in the file
\sfline{topl/Checker.java}
The checker logs a message when it detects a property violation.
Ideally, the action to take should be configurable.
During debugging, for example, an exception might be preferable to a log message.

The main components of the checker are: (1)~the data structures used to represent properties and a parser that instantiates them; (2)~the data structures used to represent the current state of the checker; and (3)~the code that handles one incoming event.

\paragraph{Parsing and AST\null.}
The parser builds ASTs based on the content of \textsf{Property.text}.
It is meant to be as straightforward as possible, and its most interesting feature is why it exists at all:
If we generate directly Java code that instantiates the ASTs, then we sometimes run over the bytecode limit on the size of methods.

The AST is more flexible than what the compiler generates.
Value guards are arbitrary boolean combinations of atomic value guards, rather than just a conjunction.

\paragraph{States.}
The checker maintains a set of active states.
Each state has a queue of events that it did not yet process.
The queue has size~$\le1$ when transitions have $\le2$~steps.
The checker does not assume a bounded transition depth, but the compiler produces only transitions with $1$~or~$2$ steps.
If all transitions would have the same depth, then one global queue of events would suffice.

Each state also has a vertex identifier and a set of bindings from automaton variables to values.
Vertex identifiers are integers; automaton variables are identified by integers; values are instances of \textsf{Object}.
Each state is produced by applying an action to another state, which is called the parent.
For error reporting, each state keeps track of its parent.
If $\ge2$~outgoing transitions are enabled for an active state, then it produces $\ge2$~active states for the next time step.
The new active states and their parent are likely to have similar sets of bindings.
This is why we use persistent (functional) sets for bindings; more precisely, we use treaps.

\paragraph{Step.}
The method \textsf{topl.Checker.check(event)} is the main loop of the checker.
One should imagine that the time is discrete, and advances when the program sends an event to the checker.

The checker does not do anything if it is inactive.
The JVM (\textbf Java \textbf virtual \textbf machine) uses selected parts of the JDK during startup.
If those parts are instrumented by the TOPL compiler, and if the TOPL checker calls back into the JDK, then the JVM crashes.
As a workaround, the checker is inactive by default.
Ideally, \textsf{toplc} should insert bytecode that activates the checker.
Currently one has to manually insert a call to \textsf{Checker.activate}, probably as the first action of the project's \textsf{main} method.
The inactive flag is not just a hack to cope with JDK instrumentation:
It is necessary to avoid recursive invocations of the checker.

The pseudocode of \textsf{topl.Checker.check} follows.

\begin{alg}
\^  $\proc{Check}(e)$
\=  $A := \emptyset$
\=  for each active state $s$
\+    if $\mathit{vertex}(s)$ does not observe $\mathit{id}(e)$
\+      insert $s$ into $A$
\=      continue from $3$
\-    push $e$ to the queue $\mathit{events}(s)$
\=    if $\mathit{events}(s)$ is shorter than the longest transition from $\mathit{vertex}(s)$
\+      continue from $3$
\1    $\mathit{skip}:=\mathsf{true}$
\=    for each transition $t$ outgoing from $\mathit{vertex}(s)$
\+      $\sigma := \mathit{store}(s)$,\quad $E := \mathit{events}(s)$
\=      for each step $\tau$ of $t$
\+        pop $\varepsilon$ from the queue $E$
\+        if $\proc{Guard}(\tau, \varepsilon, \sigma)$
\+          $\sigma := \proc{Action}(\tau, \varepsilon, \sigma)$
\-        else
\+          continue from 11
\2      $\mathit{skip}:=\mathsf{false}$
\=      insert state $(\mathit{target}(t), \sigma, E)$ into $A$
\=      if $\mathit{target}(t)$ is an error vertex, report the error
\=      if \textit{skip}
\+        $s' := s$
\=        insert state $s'$ with one event popped into $A$
\end{alg}

TODO: forget states (with strategies), runtime abstraction.

% >>>
% >>>
\section{Dacapo Properties} % <<<

This section collects properties we could possibly check with TOPL\null.

TODO: Extract those from the typestates empirical study.

\subsection{Tomcat} % <<<

Dacapo uses Tomcat version 6.0.20.

The documentation of the following classes contain constraints of the type ``event~$e$ must occur before event~$f$''.

\begin{enumerate}
\item
\textsf{java.javax.servlet.Filter}, \textsf{java.javax.servlet.Servlet}
  \begin{itemize}
  \item[$e$] \textsf{init} terminated normally (no exception)
  \item[$f$] another method of the interface is called
  \end{itemize}
\item
\textsf{java.javax.servlet.RequestDispatcher}
  \begin{itemize}
  \item[$e$] the method \textsf{m} is called with argument~\textsf{x} on all instances of~\textsf{C}
  \item[$f$] the method \textsf{n} is called with argument~\textsf{x} on some instance of~\textsf{D}
  \end{itemize}
\item
\textsf{java.javax.servlet.ServletResponse}
  \begin{itemize}
  \item[$e$] method \textsf{setCharacterEncoding}, \textsf{setContentType}, or \textsf{setLocale} is called on an object~\textsf{x}
  \item[$f$] method \textsf{getWriter}, \textsf{flush}, or \textsf{flushBuffer} is called on~\textsf{x}
  \end{itemize}
\item
\textsf{java.javax.servlet.ServletResponse}
  \begin{itemize}
  \item[$e$] method \textsf{setBufferSize} is called on an object~\textsf{x}
  \item[$f$] a method of~\textsf{w} is called, where \textsf{w} is the result of calling \textsf{getWriter} on~\textsf{x}
  \end{itemize}
\end{enumerate}

Most getters promise to return the last set value.
If there was no call to a setter, then sometimes a default value is specified in the documentation.

The following are properties of the type ``method~\textsf{m} returns a boolean that indicates whether event~\textsf{e} occurred''.

\begin{enumerate}
\item
\textsf{java.javax.servlet.ServletResponse}
  \begin{itemize}
  \item[$e$] 
  \item[$m$] \textsf{isCommitted} called on~\textsf{x}
  \end{itemize}
\end{enumerate}

PB: Could property templates be useful?
Something like parametric properties.

% >>>
\subsection{FreeBoogie} % <<<

% >>>
% >>>
\end{document}
% vim:spell errorformat=%f\:%l-%m,%f\:%l\:%m,%f\:%m
% vim:fmr=<<<,>>>:
