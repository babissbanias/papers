\documentclass[preprint]{sigplanconf} % <<<

\usepackage{amsmath}
\usepackage{amssymb}
\usepackage{amsthm}
\usepackage{graphics}
\usepackage[latin1]{inputenc}
\usepackage{microtype}  % do not remove
\usepackage{pygmentize}
\usepackage{rgalg}
\usepackage{xcolor}
\usepackage[colorlinks]{hyperref}

\RecustomVerbatimEnvironment{Verbatim}{BVerbatim}{}
\definecolor{darkblue}{rgb}{0,0,0.4}
\definecolor{verylightgray}{rgb}{0.9,0.9,0.9}
% comment the next line for printing
\hypersetup{colorlinks,linkcolor=darkblue,citecolor=darkblue,urlcolor=darkblue}
\hypersetup{
  pdftitle={A Language for Specifying Safety Temporal Properties of Object-Oriented Programs},
  pdfauthor={Dino Distefano and Radu Grigore and Rasmus Lerchedahl Petersen}}

\titlebanner{DRAFT}
\title{A Language for Specifying Safety Temporal Properties of Object-Oriented Programs}
\authorinfo{Dino Distefano \and Radu Grigore \and Rasmus Lerchedahl Petersen}{Queen Mary, University of London}{{\rm\{}ddino,rgrig,rusmus{\rm\}}@eecs.qmul.ac.uk}

\newcommand{\dinocomment}[1]{
\begin{center}
\fbox{
\begin{minipage}{3.0in}
{\bf Dino's comment:} {\it #1}
\end{minipage}}
\end{center}}



\newcommand{\N}{\ensuremath{\mathbb{N}}}
\newcommand{\pmap}{\rightharpoonup}
\newcommand{\set}[1]{\ensuremath{\mathsf{#1}}}

\theoremstyle{definition}
\newtheorem{example}{Example}

\overfullrule=5pt
\showboxdepth=10
\showboxbreadth=30
% >>>
\begin{document}
\maketitle

\begin{abstract} % <<<
In this paper we present a new specification language for temporal safety properties aimed at object-oriented languages.
The language is expressive enough to represent relationships between objects and it is designed with the goal of performing dynamic and static analysis of object-oriented software.
\end{abstract}
\category{D.2.1}{Software Engineering}{Requirements/Specifications}
\terms Languages, Verification
\keywords Safety, Temporal Properties, Object-Oriented

% >>>
\section{Introduction} % <<<
One popular class of properties addressed by many techniques in program verification is to check whether a program satisfies some specified safety property.  
Violations of this class of properties may lead to unexpected run-time errors.

Many safety properties can be expressed using the framework of {\em typestate}~\cite{strom1986} which uses finite state automata for specifying conformance or violation w.r.t. a temporal safety property.

The long term aim of our project is the automatic verification of typestate properties of Java programs of realistic size.  
The properties should be given by the user; they should not be hard-coded.
In order to achive our goal we would need
\begin{itemize}
\item A language for specifying temporal safety properties;
\item An automatic tool that can verify the properties in the language
  against Java programs.
\end{itemize}
This paper addresses the first point by introducing {\em TPL} (Temporal Property Language). 
\dinocomment{Come up with a better name or drop it!}
It has the following characteristics:
\begin{itemize}
\item It is designed for OOP. 
And since object-oriented languages like Java make heavy use of the heap, we have designed our language based on separation logic~\cite{reynolds2002} a formalist known to be effective and concise in specifying heap related properties.
\item It allows very high-level intuitive specifications which are given with an special form of automata.
\item It is designed to be used to do program analysis (both static and dynamic).
\end{itemize}  
In this paper we introduce the language and its formal semantics. 
Moreover we show how this can be used for doing dynamic checking of its properties using a simple object-based language.

% contributions
% - language for temporal properties
% - keep references (object) relationship
% - designed for oop
% - designed for doing analysis (dyn/stat)
% - High-level/simple
% Safety (obiquitus)

The paper is organized as follows. In Section~\ref{sec:example} we start with a motivating example. 
Section~\ref{sec:syntax} gives the syntax of TPL and in Section~\ref{sec:semantics} introduces its semantics. 
Section~\ref{sec:testing} describes the use of TPL for run-time checking of safety properties. 
Section~\ref{sec:related} discusses related work and our future plans. 
Finally, Section~\ref{sec:conclusions} conclude the paper.
% >>>
\section{Example}\label{sec:example} % <<<

The program in~\autoref{fig:cme} fails with an exception.
It is illegal to modify the underlying collection while iterating.

\begin{figure}\centering
\begin{Verbatim}[commandchars=\\\{\}]
\PY{k+kn}{import} \PY{n+nn}{java.util.*}\PY{o}{;}
\PY{k+kd}{public} \PY{k+kd}{class} \PY{n+nc}{IncorrectIteratorUse} \PY{o}{\PYZob{}}
  \PY{k+kd}{public} \PY{k+kd}{static} \PY{k+kt}{void} \PY{n+nf}{main}\PY{o}{(}\PY{n}{String}\PY{o}{[}\PY{o}{]} \PY{n}{args}\PY{o}{)} \PY{o}{\PYZob{}}
    \PY{n}{List}\PY{o}{<}\PY{n}{Integer}\PY{o}{>} \PY{n}{c} \PY{o}{=} \PY{k}{new} \PY{n}{ArrayList}\PY{o}{<}\PY{n}{Integer}\PY{o}{>}\PY{o}{(}\PY{o}{)}\PY{o}{;}
    \PY{n}{c}\PY{o}{.}\PY{n+na}{add}\PY{o}{(}\PY{l+m+mi}{1}\PY{o}{)}\PY{o}{;} \PY{n}{c}\PY{o}{.}\PY{n+na}{add}\PY{o}{(}\PY{l+m+mi}{2}\PY{o}{)}\PY{o}{;}
    \PY{n}{Iterator}\PY{o}{<}\PY{n}{Integer}\PY{o}{>} \PY{n}{i} \PY{o}{=} \PY{n}{c}\PY{o}{.}\PY{n+na}{iterator}\PY{o}{(}\PY{o}{)}\PY{o}{;}
    \PY{n}{Iterator}\PY{o}{<}\PY{n}{Integer}\PY{o}{>} \PY{n}{j} \PY{o}{=} \PY{n}{c}\PY{o}{.}\PY{n+na}{iterator}\PY{o}{(}\PY{o}{)}\PY{o}{;}
    \PY{n}{i}\PY{o}{.}\PY{n+na}{next}\PY{o}{(}\PY{o}{)}\PY{o}{;} \PY{n}{i}\PY{o}{.}\PY{n+na}{remove}\PY{o}{(}\PY{o}{)}\PY{o}{;} \PY{n}{j}\PY{o}{.}\PY{n+na}{next}\PY{o}{(}\PY{o}{)}\PY{o}{;}
  \PY{o}{\PYZcb{}}
\PY{o}{\PYZcb{}}
\end{Verbatim}

\caption{Running example program}
\label{fig:cme}
\end{figure}

% >>>
\section{Syntax}\label{sec:syntax} % <<<

% >>>
\section{Semantics}\label{sec:semantics} % <<<

A program's semantics is a set of event traces;
an automaton's semantics is also a set of event traces.
A program \emph{violates} a property when their sets intersect.
Both the program and the automaton are nondeterministic.

\subsection{Automaton} % <<<

Each event has a tag and carries an array of values.
\begin{align}
\set{Event}&=\bigcup_{n\in\N}\set{Tag}_n\times(n\to\set{Value})
\end{align}
(As usual, $n=\{0,1,\ldots,n-1\}$.)
The content of the undefined sets (such as \set{Value}) is not important.
We assume that these basic sets are disjoint.
Traces are finite.
\begin{align}
\set{Trace}=\bigcup_{n\in\N} n\to\set{Event}
\end{align}
Intuitively, the program outputs a trace that drives the automaton.

The automaton is defined on top of a finite multi-graph.
The automaton is an array of transitions.
Each transition has an edge and an array of labels.
Each edge has a source vertex and a target vertex.
Each label is a (guard, action) pair.
There are at least two vertices.
\begin{align}
\set{Automaton} &= \bigcup_{n\in\N} n \to \set{Transition} \\
\set{Transition} &= \set{Edge}\times \bigcup_{n\in\N} n\to\set{Label} \\
\set{Edge}&=\set{Vertex}\times\set{Vertex} \\
\set{Label}&=\set{Guard}\times\set{Action} \\
\{\mathtt{start},\mathtt{error}\}&\subseteq\set{Vertex}
\end{align}
The deterministic state contains a store of values.
A store is a finite partial map from automaton variables to values.
We also define an automaton deterministic execution state, which includes the input to be processed.
\begin{align}
\set{Store}&=\set{Variable}\pmap\set{Value} \\
\set{AState}&=\set{Vertex}\times\set{Store} \\
\set{EAState}&=\set{AState}\times\set{Trace}
\end{align}
Guards compare the values in an event with those in a store.
Actions modify the store, using values from an event.
\begin{align}
\set{Guard}&=\set{Event}\to\set{Store}\to2 \\
\set{Action}&=\set{Event}\to\set{Store}\to\set{Store}
\end{align}
The automaton deterministic step function evolves the execution state.
\begin{align}
\mathit{adStep}\in\set{EAState}\to\set{EAState}\to2
\end{align}
\begin{example}
Consider an automaton in state~$s_1$.
The input is a trace~$e_1e_2$ obtained by concatenating traces $e_1$~and~$e_2$.
The automaton must make a nondeterministic choice out of some alternatives.
A feasible alternative might be to consume $e_1$ and move to state~$s_2$.
For this, we write $(s_2,e_2)\in\mathit{adStep}(s_1,e_1e_2)$.
(We write $x\in f$ to mean that $f(x)=1$.)
\end{example}
The nondeterministic step is defined in terms of~\textit{adStep}.
\begin{align}
\mathit{anStep}&\in(\set{EAState}\to2)\to\set{EAState}\to2 \\
\mathit{anStep}\;S&=\bigcup \mathit{map}\;\mathit{adStep}\;S
\end{align}
An iterated version of \textit{anStep} is useful for defining reachable states.
We define it as the least fixed point for the following equation.
\begin{align}
\mathit{anStep}^\star\;S &= S \cup \mathit{anStep}^\star\;(\mathit{anStep}\;S)
\end{align}
Finally, we can define the set of traces described by an automaton.
\[ \{ e \mid \exists\sigma'e',\;((\mathtt{error},\sigma'),e')\in\mathit{anStep}^\star\;\{((\mathtt{start},0),e)\}\} \]
These are the traces that drive the automaton from the \texttt{start} vertex (with an empty store) to the \texttt{error} vertex.

\autoref{fig:adStep} defines \textit{adStep}.
Consider an automaton in state~$(x_1,\sigma_1)$ that processes trace~$e_1$.
If there is a transition from~$x_1$ to~$x_2$ that matches the events in a prefix~$e$ of~$e_1$, then the automaton may nondeterministically choose to perform that transition (line~10).
If no such transition exists, then the first event in $e_1$ is dropped and the automaton remains in the same state (line~11).
The actions of a transition are performed in order.
The guards are evaluated after all previous actions were performed.
A transition matches when all its guards hold.

\begin{figure}
\hbox to\hsize{\vbox{
\begin{alg}
\^  $\proc{adStep}\;((x_1,\sigma_1),e_1)\;((x_2,\sigma_2),e_2)$
\=  ~if~ $e_2$ is not a suffix of $e_1$ ~then return~ $0$
\=  $e:=\text{$e_1$ without the suffix $e_2$}$
\=  ~for each~ transition $((y_1,y_2),l)$
\+    ~if~ $(y_1,y_2)\ne(x_1,x_2) \lor \mathit{len}\;l\ne\mathit{len}\;e$ ~then continue~
\=    $\sigma:=\sigma_1$
\=    ~for each~ $k\in\mathit{len}\;l$
\+      $(g,a):=l\;k$
\=      ~if~ $\lnot(g\;(e\;k)\;\sigma)$ ~then continue~ to line 3
\=      $\sigma:=a\;(e\;k)\;\sigma$
\-    ~if~ $\sigma=\sigma_2$ ~then return~ $1$
\-  ~return~ $\mathit{len}\;e=1\land\sigma_1=\sigma_2$
\end{alg}
\smallskip
}\hfil}
\caption{One automaton step}
\label{fig:adStep}
\end{figure}

The function~\textit{adStep} is the crux of the automata semantics.
At line~5 a copy of the store is made, which is used by the following loop.
This copy is how \emph{roll-back} is built into the semantics of the automata.
If transitions always have length~$1$, then \textit{adStep} becomes simpler.
However, we could not find a desugaring into automata with transitions of length~$1$.

% >>>
\subsection{Program} % <<<

Let us now see how a simple object-oriented language produces event traces.
The state of the program is the content of the memory.
The program store (also known as stack) is similar to the automaton store.
Program variables and automaton variables live in different name-spaces.
Heaps map object addresses and field names to their values.
\begin{align}
\set{PState}&=\set{Store}\times\set{Heap}\\
\set{Heap}&=(\set{Value}\times\set{Variable})\pmap\set{Value}
\end{align}
The input is an array of values.
The program execution state keeps track of the input yet to be processed.
\begin{align}
\set{Input}&=\bigcup_{n\in\N}n\to\set{Value}\\
\set{EPState}&=\set{PState}\times\set{Input}
\end{align}
Expressions do not have side-effects.
Events are produced by the execution of simple statements.
\begin{align}
\set{Expression}&=\set{PState}\pmap\set{Value} \\
\mathit{pStep}&\in\set{PState}\to\set{Statement}\to(\set{PState}\times\set{Event})
\end{align}


The program is composed of statements.



TODO: Note that any PL that produces traces of events is suitable.
IDEA: apply to something like Haskell?

% >>>
% >>>
\section{Testing}\label{sec:testing} % <<<

% >>>
\section{Future Work}\label{sec:future} %<<<

%>>>
\section{Related Work}\label{sec:related} %<<<

%>>>
\section{Conclusions}\label{sec:conclusions} %<<<

%>>>

\softraggedright
\bibliographystyle{abbrvnat}
\bibliography{safety}
\end{document}
% vim:spell errorformat=%f\:%l-%m,%f\:%l\:%m,%f\:%m
% vim:fmr=<<<,>>>:

