\def\myarm{1cm}
\def\myangle{0}
\tikzset{
  arm/.default=1cm,
  arm/.code={\def\myarm{#1}}, % store value in \myarm
  angle/.default=0,
  angle/.code={\def\myangle{#1}} % store value in \myangle
}

% Define the myncbar to path
\tikzset{
    myncbar/.style = {to path={
        % We need to calculate a couple of coordinates to help us draw
        % the path. 
        let
            % Same as (\tikztotarget)++(\myangle:\myarm)
            \p1=($(\tikztotarget)+(\myangle:\myarm)$)
        in
            -- ++(\myangle:\myarm) coordinate (tmp)
            % Find the projection of the (tmp) coordinate
            % on the line from the target to p1
            -- ($(\tikztotarget)!(tmp)!(\p1)$)
            -- (\tikztotarget)\tikztonodes
    }}
}
% Instead of the let operation we could probably have used the 
% 'execute at begin to' option or the \pgfextra operation
\tikzset{
    stairstep/.style = {to path={
        % We need to calculate a couple of coordinates to help us draw
        % the path. 
        let
           \p1=($(\tikztostart)+(\myangle:\myarm)$),
           \p2=($(\tikztotarget)+(\myangle:\myarm)$)
       in
           % Find the projection of the (p2) coordinate
            % on the line from the start to p1
            -- ($(\tikztostart)!(\p2)!(\p1)$) coordinate (tmp)
            -- ($(tmp)!(\tikztotarget.north)!(\p2)$)\tikztonodes
    }}
}

\begin{tikzpicture}[node distance=20pt, auto]
\tikzstyle{system}=[rectangle,
                                draw=blue!80,
                                fill=blue!40,
                                inner sep=0.2cm,
                                rounded corners=10pt,
                                style=dashed,thick]
\tikzstyle{program}=[rectangle,
                                  draw=black,
                                  fill=orange!40,
                                  inner sep=0.2cm,
                                  rounded corners=5pt,
                                  style=thick,
                                  drop shadow]
\tikzstyle{data}=[rectangle,
                            draw=gray!70,
                            fill=gray!30,
                            inner sep=0.2cm,
                            rounded corners=5pt,
                            style=thick]
\node[data] (classes) {Java class files};
\node[data, right=of classes] (properties) {TOPL property files};

\node[below=of classes] (classesd) {};
\node[below=of properties] (propertiesd) {};

\node[program, below=40pt of classes] (instrumenter) {Instrumenter};
\node[program, below=40pt of properties] (genautomaton) {Automaton Generator};

\node[data, below=of genautomaton] (javaproperties) {Property.java};
\node[right=of javaproperties] (javadummy) {};
\node[data, right=of javadummy] (checker) {Checker.java};
\node[program, below=of javadummy] (javac) {Java Compiler};

\node[data, below=of javac] (classproperties) {Checker.class};
\node[left=of classproperties] (classdummy) {};
\node[data, left=of classdummy] (instrclasses) {Instrumented Java class files};
\node[program, below=of classdummy] (jvm) {JVM};

\node[above right=10pt and 20pt of instrumenter.north] (topllabel) {\emph{TOPL Compiler}};
\node[above=5pt of checker] (checkerlabel) {\emph{Property Checker}};
\begin{pgfonlayer}{background}
  \node[system, fit = (topllabel) (instrumenter) (genautomaton)] (TOPLC) {};
  \node[system, fit = (checker) (checkerlabel)] (CHECKER) {};
\end{pgfonlayer}

\path[ultra thick, ->]
(classes) edge (instrumenter)

(genautomaton) edge (instrumenter)
(instrumenter) edge (genautomaton)

(properties) edge (genautomaton)
(genautomaton)  edge (javaproperties)
(javac) edge (classproperties);

\draw[ultra thick, ->]
(instrumenter) to[myncbar, angle=-90, arm=40pt] (instrclasses);
\draw[ultra thick, ->]
(instrclasses) to[stairstep, angle=180, arm=5pt] (jvm);
%(instrclasses) -| (jvm.north);
\draw[ultra thick, ->]
 (classproperties) to[stairstep, angle=0, arm=5pt] (jvm);
\draw[ultra thick, ->]
(javaproperties) to[stairstep, angle=180, arm=5pt] (javac);
\draw[ultra thick, ->]
(checker) to[stairstep, angle=0, arm=5pt] (javac);
(\end{tikzpicture}
