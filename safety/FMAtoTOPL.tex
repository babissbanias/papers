\section{From register automata to TOPL}
%\rlp{Could still explain $\rho$ and $T$ more\ldots}
In~\cite{dblp:journals/tcs/kaminskif94} a register automaton is
defined as a six-tuple $A = \langle S, p, u, \rho, T, F\rangle$. Here $S$
is the set of vertices, $p$ is the starting vertex, $u$ is the initial
content of the registers, $\rho$ is a map from vertices to the
register they overwrite, $T$ is the transition relation and $F$ is the
set of error vertices.

$A$ thus has $n = |u|$ registers. It accepts events from some infinite
alphabet. We will assume that the symbol $\#$ is not in that alphabet and we will use it as dummy symbol useful in the translation.

Given a register automaton $A$, we will construct a TOPL property $P_A$ to simulate it. The
simulation is such that for each word $w$, $A$ accepts $w$ iff $P_A$
accepts $w' = \#^{|u|}w\#$. Thus, $w'$ is an encoding of $w$ that
caters for the differences of the two formalisms: A register automaton has an
initial state $u$ of length $n$. In order to set up the initial
state, the TOPL property needs to consume $n$ symbols, so we have to
put that many dummy symbols in front of the word. The dummy symbol at
the end is because $A$ may have many final states ($|F| > 1$) whereas a TOPL
property only has one, so we need to make an extra transition from all
the erroneous states to the actual error state.

%\begin{definition}
%Let $A=\langle S, p, u, \rho, T, F\rangle$ be an FSA
%\end{definition}
\newcommand{\Vertex}{\mathit{Vertex}}
\newcommand{\Arc}{\mathit{Arc}}
\newcommand{\sstart}{\mathit{start}}
\newcommand{\serror}{\mathit{error}}
\newcommand{\seen}{\mathit{seen}}
\newcommand{\unseen}{\mathit{unseen}}
To construct the TOPL property we have to define the set $\Vertex$
of vertices and the set $\Arc$ of labelled transitions. We do that as follows:
\[
\Vertex = S \cup \{\sstart, \serror\}
\]
that is we have the same set of vertices as $A$ plus two new
ones. $start$ is going to be the initial vertex. It will have a
transition to the start vertex $p$ of $A$ that sets up the
initial state. Thus, if $u = w_1\ldots w_n$, then $\Arc$ includes
\[
A_{\sstart} = \{\sstart \to p: (*,R_1=w_1);\ldots;(*,R_n=w_n) \}
\]
a transition of depth $n$ that assigns the letters of $u$
to $n$ automaton variables. The guard $*$ simply ignores the
event\footnote{One could also choose a guard that forces the event
to be \#.}. This transition consumes the first $n$ extra
symbols. $\Arc$ further includes
%\[
%\forall (s, i, s') \in T.\ s\to s': r_i=e, skip
%\]
\[
%A_{seen} = \bigcup_{(s, i, s') \in T} s\to s': r_i=e, skip
A_{\seen} = \{s\to s': r_i=e, skip \mid (s, i, s') \in T \}
\]
coresponding to the case where $e$ has been seen before, and
%\[
%\forall (s, i, s') \in T.\ \rho(s)=i.\ s\to s': unknown(e), R_i=e
%\]
\[
%A_{unseen} = \bigcup_{(s, i, s') \in T,\ \rho(s)=i} s\to s': unknown(e), R_i=e
A_{unseen} = \{s\to s': g_\nu(e), R_i=e \mid (s, i, s') \in T\ \land\ \rho(s)=i \}
\]
where $g_\nu(e) = r_1 \neq e \land \ldots \land r_n \neq e$,
corresponding to the case where $e$ has not been seen (or has been
forgotten). In this case, the relevant register is
updated. Finally $\Arc$ includes
%\[
%\forall s\in F.\ s\to error: *, skip
%\]
\[
%A_{error} = \bigcup_{s\in F} s\to error: *, skip
A_{\serror} = \{s\to \serror: *, skip \mid s\in F \}
\]
to send the vertices in $F$ to the error state. This transition
consumes the final extra symbol.

Formally, we can define
\[
\Arc = A_{\sstart} \cup A_{\seen} \cup A_{\unseen} \cup A_{\serror}.
\]

The question of going in the other direction is harder because TOPL
properties can have labels with depth more than one. We looked for some
time for a translation that would rid us of such labels, but found
nothing satisfactory, which is why there are still deep labels in the
intermediate form targeted by desugaring (see \autoref{sec:TOPLC}). We leave this
question open here.