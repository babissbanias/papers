\section{From register automata to TOPL}
\rlp{Some definitions and theorems?}
\dd{Maybe explain the components of a register automaton} 
Given a register automaton~\cite{dblp:journals/tcs/kaminskif94} $A=\langle S, p, u, \rho, T,
F\rangle$, we will construct a TOPL property $P_A$ to simulate it. The
simulation is such that for each word $w$, $A$ accepts $w$ iff $P_A$
accepts $w' = \#^{|u|}w\#$. \dinocomment{what does it mean? what are the $\#$?} Thus, $w'$ is an encoding of $w$ that
caters for the differences of the two formalisms: A register automaton has an
initial state $u$ of length $|u|$. In order to set up the initial
state, the TOPL property needs to consume $|u|$ symbols, so we have to
put that many dummy symbols in front of the word. The dummy symbol at
the end is because $A$ may have many final states ($|F| > 1$) whereas a TOPL
property only has one, so we need to make an extra transition from all
the erroneous states to the actual error state.

%\begin{definition}
%Let $A=\langle S, p, u, \rho, T, F\rangle$ be an FSA
%\end{definition}
\newcommand{\Vertex}{\mathit{Vertex}}
\newcommand{\Arc}{\mathit{Arc}}
To construct the TOPL property we have to define the set $\Vertex$
of vertices and the set $\Arc$ of labelled transitions. We do that as follows:
\[
\Vertex = S \cup \{start, error\}
\]
that is we have the same set of vertices as $A$ plus two new
ones. $start$ is going to be the initial vertex. It will have a
transition to the start vertex $p$ of $A$ that sets up the
initial state. Thus, if $u = w_1\ldots w_n$, then Arc includes
\[
A_{start} = start \to p: (*,R_1=w_1);\ldots;(*,R_n=w_n)
\]
a transition of depth $n = |u|$ that assigns the letters of $u$
to $n$ automaton variables. The guard $*$ simply ignores the
event\footnote{One could also chose a guard that forces the event
to be \#.}. This transition consumes the first $n$ extra
symbols. $\Arc$ further includes
%\[
%\forall (s, i, s') \in T.\ s\to s': r_i=e, skip
%\]
\[
A_{seen} = \bigcup_{(s, i, s') \in T} s\to s': r_i=e, skip
\]
coresponding to the case where $e$ has been seen before, and
%\[
%\forall (s, i, s') \in T.\ \rho(s)=i.\ s\to s': unknown(e), R_i=e
%\]
\[
A_{unseen} = \bigcup_{(s, i, s') \in T,\ \rho(s)=i} s\to s': unknown(e), R_i=e
\]
where $unknown(e) = r_1 \neq e \land \ldots \land r_n \neq e$,
coresponding to the case where $e$ has not been seen (or has been
forgotten). In this case, the relevant register is
updated. Finally $\Arc$ includes
%\[
%\forall s\in F.\ s\to error: *, skip
%\]
\[
A_{error} = \bigcup_{s\in F} s\to error: *, skip
\]
to send the vertices in $F$ to the error state. This transition
consumes the final extra symbol.

Formally, we can define
\[
\mathrm{Arc} = A_{start} \cup A_{seen} \cup A_{unseen} \cup A_{error}.
\]
