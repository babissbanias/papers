\def\pmap{\rightharpoonup}
\centerline{\bf mop}

\medskip\noindent This note summarizes the article {\it Semantics and Algorithmics for Parametric Monitoring\/}.
\medskip

The model of computation is a set of automaton instances running in parallel.
The automata are deterministic and may have an infinite number of states.
Each instance sees only a slice of the trace.
There is a set~$X$ of {\it parameters\/}, a set~$V$ of {\it values\/}; conceptually, there is one automaton instance running for each binding $\theta:X\pmap V$.
Each event has a name $e : E$ and a binding $\theta':X\pmap V$.
The automaton instance for~$\theta$ sees exactly the events carrying bindings $\theta'\le\theta$.
The order is the usual order on partial maps:
$\theta'\le\theta$ means that the two maps agree on the domain of~$\theta'$, which is no bigger than the domain of~$\theta$.

An automaton has a possibly infinite set~$S$ of states and a transition function $\sigma:(S\times E)\to S$.
Notice that the transition function sees only the name of the event, not the bindings.
For each automaton instance~$\theta$, a different set~$F\subseteq S$ is accepting.

The paper culminates with an online algorithm for slicing traces.
Its simple version is straightforward, but I wouldn't say simple.
To get the intuition behind it, please consider the following problem.
Suppose we have some hash function for sequences of integers that may be computed online (one integer at a time), such as $h({\bf x}y)=31h({\bf x})+y$.
We are given a sequence~${\bf x}$ of integers and we are asked to process it one integer at a time and construct a data structure that can answer queries of the form:
``What is the hash of the subsequence of integers $\le k$?''
Preprocessing $O(n\lg\lg n)$ time, these queries can be answered in $O(\lg\lg n)$ time, where $n=|{\bf x}|$.
TODO: Continue here.

\bye
% vim:wrap:linebreak:fmr=<<<,>>>:nosi:spell:
